El siguiente trabajo de graduación tiene como objetivo el diseño y la implementación, enfocándose en el módulo analógico, de cuatro tableros de control industrial, esto con fines didácticos para el uso en la Universidad del Valle de Guatemala. Para el diseño del mismo se utilizó el software Autodesk Electrical, en donde de igual manera se generaron diagramas unifilares respectivos. Para su implementación física se utilizaron componentes ya disponibles en la universidad. De igual forma fue necesario realizar la estimación y compra de materiales como cables y rieles para su correcta implementación.

Ya que el módulo analógico fue el énfasis en el trabajo, la implementación física fue realizada individualmente en cada tablero. Los módulos analógicos utilizados como entrada fueron los SIEMENS 6ES7331-7KF02-0AB0. Las salidas fueron lo los módulos SIEMENS 6ES7332-5HF00-0AB0. Finalmente, para garantizar el buen funcionamiento del tablero se procedió a realizar pruebas en los mismos. Estas consistieron en llevar a cabo laboratorios de la clase Instrumentación y Automatización Industrial y corroborar que los resultados fueran satisfactorios.