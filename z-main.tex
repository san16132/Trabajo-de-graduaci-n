% Se pre-carga la información del estudiante sólo para poder emplear el macro de
% selección de versión (digital o impresa)
\input{0-datos_estudiante}

\ifdefined\printver
    \documentclass[11pt, letterpaper, twoside, openright]{report}
\else
    \documentclass[11pt, letterpaper]{report}
\fi

% Eliminar la opción de twoside y openright si se desea generar la versión
% digital del documento en lugar de la versión impresa
%\documentclass[11pt, letterpaper, twoside, openright]{report}
\usepackage[spanish, es-nodecimaldot, es-noquoting]{babel}
% cambiar a spanish, mexico si se quiere emplear tabla en lugar de cuadro
\selectlanguage{spanish}
\usepackage[utf8]{inputenc}
\usepackage[T1]{fontenc}

\title{Plantilla para protocolo de trabajos de graduación IE-MT 2020v1}
\author{MSc. Miguel Zea}
\date{\today}

% Información del estudiante en el archivo datos_estudiante.tex
\input{0-datos_estudiante}

% ==============================================================================
% DEFINICIÓN DE PAQUETES
% ==============================================================================
\usepackage{xcolor}
\usepackage{amsfonts}
\usepackage{amsmath}
\usepackage{amssymb}
\usepackage{amsthm}
\usepackage{amsfonts}
\usepackage{mathtools}
\usepackage{graphicx}
\usepackage{xfrac}
\usepackage{float}
\usepackage{mathtools}
\usepackage[hypertexnames=false]{hyperref}
% \usepackage{bookmark}
\usepackage[font=small]{caption}
\usepackage{subcaption}
%\usepackage{csquotes}
\usepackage{xpatch}
\usepackage{emptypage}
\usepackage{hyphenat}
\usepackage{fancyhdr}
\usepackage{float}
\usepackage[backend=biber, style=ieee]{biblatex}
\ifdefined\usarAPA 
    \usepackage[backend=biber, style=apa]{biblatex}
\fi
\addbibresource{m-bibliografia.bib}

\usepackage[percent]{overpic}

\usepackage{chngcntr}

% ==============================================================================
% MÁRGENES Y FORMATO GENERALES
% ==============================================================================
\usepackage[top=1in, left=1.5in, right=1in, bottom=1in]{geometry}
%Options: Sonny, Lenny, Glenn, Conny, Rejne, Bjarne, Bjornstrup
\usepackage[Sonny]{fncychap}

% ==============================================================================
% DEFINICIONES DE LA PLANTILLA
% ==============================================================================
\definecolor{uvg-green}{RGB}{17,71,52}
\newcommand{\defaultparformat}[1]{
	{\setlength{\parskip}{2ex}
     \input{#1}}
}
\ifdefined\capsecuvg
	\renewcommand\thechapter{\Roman{chapter}}
    \renewcommand\thesection{\Alph{section}}
	\renewcommand\thesubsection{\arabic{subsection}}
    \renewcommand\thesubsubsection{\alph{subsubection}}
\fi
\counterwithout{figure}{chapter}
\counterwithout{table}{chapter}
\counterwithout{equation}{chapter}

\newcommand{\blankpage}{
\newpage
\thispagestyle{empty}
\mbox{}
\newpage
}
% ==============================================================================

% Comandos definidos por el usuario en el archivo comandos_usuario.tex
\input{2-paquetes_y_comandos_usuario}

% ==============================================================================
% CUERPO DEL TRABAJO
% ==============================================================================
\pagestyle{headings}
\begin{document}

% Se define la carpeta de imágenes
\graphicspath{{figuras/}}

% ==============================================================================
% PRIMERAS PÁGINAS (Carátulas más hojas de guarda)
% ==============================================================================

\newpage
\cleardoublepage\phantomsection
\pagecolor{white}
\color{black}
\setcounter{page}{1}
\pagenumbering{roman}
\thispagestyle{empty}
\begin{center}
	\LARGE UNIVERSIDAD DEL VALLE DE GUATEMALA\\
	\LARGE Facultad de \uvgfacultad \\[0.75cm]
\end{center}
\begin{figure}[h]
	\begin{center}
	\includegraphics[height=5.5 cm]{plantilla/escudoUVGnegro.eps}
	\vspace{0.5in}
	\end{center}
\end{figure}
\begin{center}
	\Large \textbf{\nohyphens{\titulotesis}} \\
	%\LARGE \textbf{\titulotesis} \\
	\vfill
	\Large \nohyphens{Protocolo de trabajo de graduación presentado por \nombreestudiante, \ estudiante de \uvgcarrera} \\
	\vfill
	\large Guatemala, \\
	\vspace{1em}
	\anoentrega
\end{center}


\pagestyle{plain}

% ==============================================================================
% CONTENIDO DEL TRABAJO
% ==============================================================================
\newpage
\cleardoublepage
\pagenumbering{arabic}
\setcounter{page}{1}

\section*{Resumen}
\defaultparformat{b-resumen}

\section*{Antecedentes}
\defaultparformat{e-antecedentes}

\section*{Justificación}
\defaultparformat{f-justificacion}

\section*{Objetivos}
\defaultparformat{g-objetivos}

\section*{Marco teórico}
\defaultparformat{i-marco_teorico}

\section*{Metodología}
\defaultparformat{p-metodologia}

\section*{Cronograma de actividades}
\defaultparformat{q-cronograma}

\section*{Índice preliminar}
\defaultparformat{r-indice_preliminar}

\section*{Referencias}
\printbibliography[heading=none]


\end{document}