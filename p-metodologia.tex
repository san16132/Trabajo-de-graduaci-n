Para poder llevar a cabo los objetivos definidos anteriormente se realizaron los planos unifilares del módulo digital, analógico, de 24 V DC, de alta potencia incluyendo el variador de frecuencias y finalmente el módulo del CPU, breakers, HMI y conexión del Profibus. Estos diagramas unifilares se implementaron utilizando como referencia los realizados por la empresa ESINSA. El uso de AutoCAD Electrical fue necesario para la realización de los mismos bajo la norma IEC 60617. Posteriormente se llevo a cabo la investigación de la norma IEC 60446, la cual indica las normas a la hora de escoger el color del cableado según su aplicación. 

Posteriormente se realizaron mediciones para estimar la cantidad de cable necesario para conectar el módulo de entrada y salida analógico en la configuración actual, propuesta por ESINSA. De la misma forma se realizó la estimación de cable necesario en la nueva configuración, en donde se estima un ahorro de cable del 24 por ciento. de la misma forma se realizó un listado de componentes necesarios para la instalación del sistema, siendo considerados las borneras, canaletas, rieles de fijación, cables y separadores de borneras. 

Para obtener una representación más intuitiva del tablero industrial se llevó a cabo el diseño del mismo en Autodesk Inventor. Este se realizó de la forma más realista posible, incluyendo los cables y demás componentes del tablero. Los modelos CAD de los módulos analógicos, digitales, CPU, HMI y del variador de frecuencia fueron obtenidos a través de internet, siendo estos de uso libre con los respectivos créditos a sus autores. 

Finalmente se realizó la instalación de la entrada y salida analógica antes mencionada. La implementación en los 4 tableros se llevó acabo según el cronograma. Para corroborar el buen funcionamiento se llevo a cabo un laboratorio de la clase de Instrumentación Industrial 1, impartido en la Universidad del Valle de Guatemala.