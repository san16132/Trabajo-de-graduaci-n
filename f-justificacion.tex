Ante el aumento de capacidad del departamento de electrónica, mecatrónica y biomédica por la construcción del nuevo edificio de la Universidad del Valle de Guatemala, llamado CIT (Centro de Innovación y Tecnología), habrá más demanda de equipo en los laboratorios, por lo que los seis tableros de control industrial actuales no serán suficientes. Por medio de este trabajo graduación, junto a otros compañeros, se diseñó y se implementó cuatro tableros de control industrial, para tener un total de 10. Con una mayor cantidad de tableros más personas podrán trabajar en el laboratorio al mismo tiempo, y los grupos de trabajo en los mismos podrán ser más reducidos. Esto se traduce en una mejor experiencia y aprendizaje en la clase de Instrumentación y Automatización Industrial I y II. 